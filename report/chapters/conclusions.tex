\pagebreak
\section{Podsumowanie}
  \begin{itemize}
    \item{Podczas rozgrywek z agentem losowym, algorytm MCTS był w stanie wygrać 70 \% rozgrywek, w przypadku gdy rozpoczynał grę 
          oraz 50 \% rozgrywek, gdy agent rozpoczynał.}
    \item{Najlepsze wyniki osiągał algorytm MCTS w pojedynku z agentem kontrolującym -- był w stanie wygrać 80 \% oraz 100 \% 
          rozgrywek, gdy odpowiednio rozpoczynał grę oraz gdy agent rozpoczynał.}
    \item{Z kolei najgorzej sobie radził algorytm MCTS z agentem agresywnym. Wygrywał tylko pojedyncze gry, co przełożyło się na
          następujący wynik: 90 \% przegranych w przypadku rozpoczynania gry oraz 100 \% przegranych w przeciwnym przypadku.}
    \item{Parametrem, który wpływał na jakość osiąganych wyników, był limit czasowy nakładany na czas działania algorytmu / 
          podejmowanie decyzji w ramach pojedynczej tury. Im większy czas, tym więcej symulacji można przeprowadzić. Należy
          jednak pamiętać, aby nie ustawić zbyt wysokiej wartości, ponieważ to bezpośrednio wpływa na długość całej rozgrywki.
          Należało zatem dobrać parametr tak, aby przeprowadzić eksperyment w rozsądnym czasie.}
    \item{W przypadku części wykresów, można zauważyć niską liczbę przeprowadzonych symulacji (pojedyncze kropki). Wynika to
          z faktu, że dla części stanów gry istnieje duża licza możliwych ruchów i czas ich wyznaczania jest dość wysoki. Im więcej
          zatem czasu zostanie przeznaczone na generowanie ruchów, tym mniej kroków algorytmu można przeprowadzić.}
    \item{W celu uniknięcia sytuacji, w której algorytm nie może wykonać pojedynczego przebiegu ze względu na czas generowania
          dostępnych ruchów, został nałożony 5 sekundowy limit czasowy.}
    \item{Często również można zaobserwować, że maksymalna, średnia oraz mediana głębokości liści osiągają tę samą wartość (równą 2).
          Następuje to w sytuacji, w której z aktualnego stanu gry można doprowadzić do zakończenia gry (zabicia jednego z graczy).
          Pomimo takiej możliwości algorytm MCTS nie zawsze podejmuje decyzje, które kończą rozgrywkę (może to wynikać z dużej liczby
          możliwych ruchów i braku czasu na ich zbadanie / przejrzenie).}
  \end{itemize}
