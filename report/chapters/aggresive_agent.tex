\subsection{Agent agresywny}

Celem agenta agresywnego jest maksymalizacja obrażeń zadawanych przeciwnikowi. W pierwszej kolejności aplikowane są wszystkie ruchy atakujące przeciwnika. Następnie podejmowana jest próba kontroli pola gry, pod warunkiem, że ocena pola gry agenta jest gorsza niż ocena dla stanu planszy przeciwnika -- wybierana jest najlepsza akcja ze zbioru rzucania czarów oraz układania minionów na planszy. Kryterium wybrania najlepszej akcji to parametr \emph{aggressive\_rate}, zdefiniowany dla każdej karty, natomiast ocena pola gry oparta jest liczbę punktów życia oraz ataku dla każdego z minionów, znajdującego się na polu gry danego gracza.

\begin{figure}[H]
	\begin{minted}{python}
def play_turn():
	while True:
		possible_actions = get_possible_actions()
		
		if empty(possible_actions)
			end_turn()
		
		# Attacking enemy hero
		minion_attacks = get_minion_attacks(possible_actions)
		
		if not_empty(minion_attacks):
			for action in minion_attacks:
				if minion_attacks_oponnent(action)
					perform_action(action)
				
					if is_dead(enemy):
						end_turn()
			
		# Checking field	
		spell_plays = get_play_spells(possible_actions)
		minion_puts = get_minion_puts(possible_actions)
				
		if not_empty(spell_plays) or not_empty(minion_puts):
			if score_field(enemy) >= score_field(agent):
				best_action = choose_best_action(spell_plays, minion_puts)
				perform_action(best_action)
			else:
				end_turn()
	\end{minted}
	\caption{Pseudokod działania agenta agresywnego.}
\end{figure}