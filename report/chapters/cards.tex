\pagebreak
\section{Karty}

Poniżej wymienione zostały karty, które użyto przy implementacji gry. Każda składa się z atrybutów:

\begin{itemize}
	\item nazwa,
	\item koszt,
	\item wskaźnik kontrolowania,
	\item wskaźnik agresywności.
\end{itemize}

Dodatkowo karty minionów zawierają pola takie jak:

\begin{itemize}
	\item zdrowie,
	\item atak,
	\item efekt uboczny (w niektórych kartach go nie ma).
\end{itemize}

Karty czarów mają jedno dodatkowe pole: efekt.

\begin{table}[H]
	\centering
	\begin{tabular}{|c|c|c|c|c|c|}
		\hline
		\textbf{Nazwa karty} & \textbf{Koszt} & \textbf{Zdrowie} & \textbf{Atak} & \textbf{W. kontrolowania} & \textbf{W. agresywności} \\
		\hline
		Enchanted Raven & 1 & 2 & 2 & 2 & 2 \\
		\hline
		Spider Tank & 3 & 4 & 3 & 4 & 3 \\
		\hline
		Ice Rager & 3 & 2 & 5 & 2 & 5 \\
		\hline
		Worgen Greaser & 4 & 3 & 6 & 3 & 6 \\
		\hline
		Am'gam Rager & 3 & 5 & 1 & 5 & 1 \\
		\hline
	\end{tabular}
	\caption{Lista kart minionów, bez efektów ubocznych.}
\end{table}


\begin{table}[H]
	\centering
	\begin{tabular}{|c|c|c|c|c|c|c|}
		\hline
		\textbf{Nazwa karty} & \textbf{Koszt} & \textbf{Zdrowie} & \textbf{Atak} & \textbf{W. kontr.} & \textbf{W. agr.} & \textbf{Efekt uboczny} \\
		\hline
		Cult Apothecary & 5 & 4 & 4 & 4 & 4 & restore\_health\_for\_minions \\
		\hline
		Eadric the Pure & 7 & 7 & 3 & 7 & 3 & reduce\_enemy\_minions\_attack\_points \\
		\hline
	\end{tabular}
		\caption{Lista kart minionów, z ubocznych.}
\end{table}

\begin{table}[H]
	\centering
	\begin{tabular}{|c|c|c|c|c|}
		\hline
		\textbf{Nazwa karty} & \textbf{Koszt} & \textbf{W. kontrolowania} & \textbf{W. agresywności} & \textbf{Efekt} \\
		\hline
		Flamestrike & 7 & 5 & 5 & deal\_damage\_to\_enemy\_minions \\
		\hline
		Sprint & 7 & 0 & 0 & draw\_cards \\
		\hline
		Consecration & 4 & 7 & 7 & deal\_damage\_to\_all\_enemies \\
		\hline
	\end{tabular}
		\caption{Lista kart czarów.}
\end{table}