\subsection{Agent kontrolujący}

W pierwszym etapie tury agenta kontrolującego wybierane są akcje ze zbiorów: rzucania czarów, ustawiania stronników oraz atakowania przez nich stronników przeciwnika. Mają one na celu uzyskanie lepszego pola gry niż przeciwnik i wybierane są w kolejności określonej przez parametr \emph{controlling\_rate}. Akcje są wykonywane do momentu wykorzystania puli ruchów wspomnianych wcześniej typów lub uzyskania lepszego pola. 

\noindent Kolejnym etapem tury agenta jest wykonywanie akcji, których celem jest atak na stronników przeciwnika lub niego samego, gdy nie ma on żadnych stronników.

\begin{figure}[H]
	\begin{minted}{python}
def play_turn():
	# Checking field
	while True:
		possible_actions = get_possible_actions()
		
		if empty(possible_actions):
			end_turn()
	
		minion_minion_attacks = get_minion_to_minion_attacks(possible_actions)
		spell_plays = get_play_spells(possible_actions)
		minion_puts = get_minion_puts(possible_actions)
		
		checking_field_actions = (minion_minion_attacks, spell_plays, minion_puts)
		if not_empty(checking_field_actions):
			if score_field(enemy) >= score_field(agent):
				best_action = choose_best_action(checking_field_actions)
				perform_action(best_action)
			else:
				break
		else:
			break
			
	# Attacking
	while True:
		minion_plays = get_minion_attacks(get_possible_actions())
		
		if empty(minion_plays):
			end_turn()
	
		if has_not_any_minions_on_field(enemy):
			for action in get_minion_to_enemy_attacks(minion_plays):
				perform_action(action)
			
				if is_dead(enemy):
					end_turn()
					
		else:
			actions = get_minion_to_minion_attacks(minion_plays):
			perform_action(actions[0])
			
	\end{minted}
	\caption{Pseudokod działania agenta kontrolującego.}
\end{figure}