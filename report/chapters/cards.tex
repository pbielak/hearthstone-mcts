%\pagebreak
\section{Karty}

Każda z kart, które wykorzystana została przy implementacji gry, posiada następujące atrybuty:

\begin{itemize}
	\item nazwa,
	\item koszt,
	\item wskaźnik kontrolowania,
	\item wskaźnik agresywności.
\end{itemize}

Karty stronników oraz magiczne posiadają dodatkowe pola, która zostaną opisane w dalszej części sprawozdania.

\subsection{Stronnicy}

Karty stronników zawierają dodatkowe pola, takie jak:

\begin{itemize}
	\item zdrowie,
	\item atak.
\end{itemize}

Niektóre z kart stronników zawierają dodatkowy atrybut: efekt uboczny. Informuje on o tym, jaki efekt pojawia się po umieszczeniu karty danego stronnika na polu gracza. Karty stronników mogą zostać wykorzystane przeciwko przeciwnikowi oraz jego stronnikom.

\begin{table}[H]
	\centering
	\begin{tabular}{|c|c|c|c|c|c|}
		\hline
		\textbf{Nazwa karty} & \textbf{Koszt} & \textbf{Zdrowie} & \textbf{Atak} & \textbf{W. kontrolowania} & \textbf{W. agresywności} \\
		\hline
		Enchanted Raven & 1 & 2 & 2 & 2 & 2 \\
		\hline
		Spider Tank & 3 & 4 & 3 & 4 & 3 \\
		\hline
		Ice Rager & 3 & 2 & 5 & 2 & 5 \\
		\hline
		Worgen Greaser & 4 & 3 & 6 & 3 & 6 \\
		\hline
		Am'gam Rager & 3 & 5 & 1 & 5 & 1 \\
		\hline
	\end{tabular}
	\caption{Lista kart stronników, bez efektów ubocznych.}
\end{table}


\begin{table}[H]
	\centering
	\begin{tabular}{|c|c|c|c|c|c|c|}
		\hline
		\textbf{Nazwa karty} & \textbf{Koszt} & \textbf{Zdrowie} & \textbf{Atak} & \textbf{W. kontr.} & \textbf{W. agr.} & \textbf{Efekt uboczny} \\
		\hline
		Cult Apothecary & 5 & 4 & 4 & 4 & 4 & restore\_health\_for\_minions \\
		\hline
		Eadric the Pure & 7 & 7 & 3 & 7 & 3 & reduce\_enemy\_minions\_attack\_points \\
		\hline
	\end{tabular}
	\caption{Lista kart stronników, z efektami ubocznymi.}
\end{table}

Opis efektów ubocznych:

\begin{itemize}
	\item \textbf{restore\_health\_for\_minions} -- graczowi, który umieszcza kartę stronnika, przywracane są punkty zdrowia. Wartość o jaką zwiększany jest stan zdrowia to wynik przemnożenia liczby stronników przeciwników przez 2. Wartość punktów zdrowia po aktualizacji nie będzie większa niż maksymalny poziom punktów zdrowia (domyślnie 20);
	\item \textbf{reduce\_enemy\_minions\_attack\_points} -- zmniejsza siłę ataków wszystkich stronników przeciwnika do wartości 1.
\end{itemize}

\subsection{Czary}

Karty czarów mają jedno dodatkowe pole: efekt. Działa on podobnie jak efekt uboczny w kartach stronników, należy jednak pamiętać, że karta czaru jest zagrywana natychmiastowo po wybraniu, po czym znika z gry. Ten typ kart może zostać użyty jedynie dla całej gry. Nie ma możliwości wyboru celu, tak jak w poprzednim typie kart.

\begin{table}[H]
	\centering
	\begin{tabular}{|c|c|c|c|c|}
		\hline
		\textbf{Nazwa karty} & \textbf{Koszt} & \textbf{W. kontrolowania} & \textbf{W. agresywności} & \textbf{Efekt} \\
		\hline
		Flamestrike & 7 & 5 & 5 & deal\_damage\_to\_enemy\_minions \\
		\hline
		Sprint & 7 & 7 & 7 & draw\_cards \\
		\hline
		Consecration & 4 & 7 & 7 & deal\_damage\_to\_all\_enemies \\
		\hline
	\end{tabular}
		\caption{Lista kart czarów.}
\end{table}

Opis efektów:

\begin{itemize}
	\item \textbf{deal\_damage\_to\_enemy\_minions} -- każdy ze stronników przeciwnika otrzymuje obrażenia w postaci odjęcia 4 punktów zdrowia. Gdy liczba punktów zdrowia będzie równa lub mniejsza od 0, stronnik przeciwnika zostaje usunięty z gry;
	\item \textbf{draw\_cards} -- z talii kart aktualnego gracza dociągane są 4 karty. W przypadku, gdy w talii nie ma już kart, gracz traci liczbę punktów równoważną liczbie prób pociągnięcia kart z pustej talii;
	\item \textbf{deal\_damage\_to\_all\_enemies} -- każdy stronnik przeciwnika oraz on sam traci 2 punkty zdrowia.
\end{itemize}